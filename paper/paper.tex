\documentclass[letterpaper,10pt,onecolumn]{article}

\usepackage{setspace} 
\singlespacing
\usepackage{graphicx}                                        
\usepackage{amssymb}                                         
\usepackage{amsmath}                                         
\usepackage{amsthm}
\usepackage{url}         
\usepackage{tabu}                             

\usepackage{geometry}
\geometry{textheight=9.5in, textwidth=7in}
\parindent = 0.0 in
\parskip = 0.1 in
\title{Imperative Programming with Variational Effects}
\author{Alex Grasley}
\begin{document}

\maketitle

\section{Introduction}

At its most basic level, all computation can be viewed as a function from a set of inputs to some
output. For example, consider a function that computes the cost of a car given certain options such
as color, engine size, and warranty. In Haskell we might denote the type of this function as
\texttt{Color -> EngineSize -> Warranty -> Cost}. Given this function, suppose we want to present
a customer with the cost of a car at different warranty levels. We must then run the function a number of
times equal to the number of warranty levels, varying the warranty parameter while keeping the other
parameters constant.

This situation is not ideal. The result is that we repeat work calculating cost based off of color and engine
size every time we try a new warranty variant. Any calculations that are shared across executions are
lost under this model. We would also like a way to explicitly represent in our language what it is 
we are doing when we perform multiple executions and collect the results. That is, we would like a way
to explicitly represent variation as a first-class construct in our language. We can imagine a function
of type \texttt{Color -> EngineSize -> V Warranty -> V Cost} where \texttt{V Warranty} represents a
variational warranty which maps to a variational cost as the result of the function. Because we can
now model variational data explicitly, we can also take advantage of any sharing that occurs between
executions, only branching our execution where we encounter variation.

% TODO introduce choice calculus

\section{Motivation}





\end{document}